\section{Hardware module}\label{sec:overview}
The architecture of this differential system can be separated in several functional blocks: the DAC, mixer, level shifter and amplifier.
PLAATJE
Whereby the DAC translates a 15 bit unary coded digital signal with a sample frequency of maximum 500MHz to an analogue signal. There is chosen for unary coding to increase linearity compared to binary coded signals. One of the reasons is that in the creation process of the transistors, it is more precise to make to transistors of the same size, than to make one with exactly two times the size. This means that the maximum theoretical SNR is 25.8dB.
SNR = 6.02 x n + 1.761 = 25.841dB 
The mixer works with a square wave, 2 GHz local oscillator (LO) frequency. The mixer can be divided in two types. The p-type and the n-type, where the p-type consist out of a flip-flop and a NAND and the n-type out of a flip-flop and a NOR.
The level shifter is responsible for the change of the 1.1V power supply for the thin-oxide transistors to the 5V power supply for the thick-oxide transistors.
The final functional block of this design, the amplifier, should provide enough output power to drive the antenna. The specified output current is 50mA, which means that the output power on the 50 ohm matched antenna should be 20.97dBm.
The aim is to get a IMD3 of at least … and an efficiency of … 
Sensitivity…
