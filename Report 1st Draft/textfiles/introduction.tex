\section{Introduction} \label{sec:introduction}
In telecommunication applications there is the need to translate digital information to an analogue signal that can be send by an antenna. Nowadays many architectures exist that accomplish this, where most are derived from a more traditional architecture (see Fig. ~\ref{fig:traditional}a). One of these architectures combines the DAC, mixer and power amplifier (PA) in a CMOS structure (Power-Mixing-DAC) Fig.~\ref{fig:traditional}b. Such a solution could be useful for several systems. Mainly for systems that require high speed, low power consumption or lack of sufficient available space for a PCB. One example for such a system is the Wi-Fi connection in a mobile phone.
\begin{figure}[h]
\includegraphics[width=0.5\textwidth]{traditional.PNG}
\caption{A traditional transmitter stage (a) and the combined Power-Mixing-DAC (b). ~\cite{powerdac}}
\label{fig:traditional}
\end{figure} 
Because the Power-Mixing-DAC is fitted in a CMOS structure, it requires less space than a traditional system. This allows the system to be faster because there are less parasitic capacitances between the structures. Also the need to match to 50 ohm between these structures becomes superfluous, because all components can be tuned to each other. No particular standard has to be followed to combine components to those of other manufacturers therefore the system requires less components. Moreover the use of large capacitors or inductors is not possible, because these require too much space to create in silicon. Normally these are placed on the PCB.
This paper shows the design and analysis of a Power-Mixing-DAC in a 65nm CMOS dummy technology, with as goal to mix a signal with maximum bandwidth of 500MHz with a local oscillator (LO) with a frequency of 2 GHz and reach a 20.97dBm output power and a IMD3 > 30dBc. The design and analysis is divided in two steps. The first step is to create the Power-Mixing-DAC with ideal components, to determine the theoretical boundaries, this design is used as reference. The second step is to replace the ideal components with (non-ideal) transistors, where the parameters are designed to come as close as possible to the ideal-component design.
