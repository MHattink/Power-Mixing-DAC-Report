\section{Introduction} \label{sec:introduction}
In telecommunication applications there is the need to translate digital stored information to an analogue signal that can be send by an antenna. Now a days there are many architectures to accomplish this, where most are distracted from a more traditional architecture. (figure 1.a). One of these architectures combines the DAC, mixer and power amplifier (PA) in a CMOS structure (hereafter Power-Mixing-DAC)(figure1.b). Such a solution could be useful for several systems. Mainly for systems that require high speed, low power consumption or lack of sufficient available space for a PCB. One example for such a system is the Wi-Fi connection in a mobile phone.
PLAATJE
Because the system is fitted in a CMOS structure, it requires less space, which leads to a more compact solution. This allows the system to be faster, because there are less parasitic capacitances between the structures. Also the need of matching to 50 ohm between these structures becomes superfluous, therefore the system requires components. Using such a combined system solution has as disadvantage that it is hard to generate high power, because power leakage generates heat which can damage the transistors. Moreover the use of large capacitors or inductors is not possible, because these require too much space to create in silicon. Normally these are placed on the PCB.
The paper shows the design and analysis of a Power-Mixing-DAC in a 45nm CMOS dummy technology, with as goal a local oscillator (LO) frequency of 2 GHz, a maximum bandwidth of 500MHz, a 20.97dBm output power and a IMD3 > 30dBc. The design and analysis is divided in two steps. The first step to create the Power-Mixing-DAC with ideal components, to know where the theoretical boundaries are, this design is used as reference. The second design is composed with transistor components, where the design parameters are chosen to come as close as possible to ideal-component design.  



